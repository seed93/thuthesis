

\chapter{提取特征}
\label{chap:feature}
本文采用的是基于特征的方法,如何用合适的特征描述图像显得至关重要。首先,选取的特征需要能很好地刻画人物的姿态信息,即相似姿态具有相似的特征,不同姿态的特征的区别也要足够明显,尽可能让特征只包含姿态信息,不会因为光照、图片尺寸、衣服、人物体型的改变受到剧烈变化。其次,特征的维数不能太高,否则会增加学习和预测过程中的计算量。同时提取特征本身的计算量也要加以限制。
经过调查发现,目前最好的姿态估计算法~\cite{ramanan2007learning}~\cite{yang2011articulated}~\cite{tian2012exploring} 大都采用了HOG(Histogram of Oriented Gradient,方向梯度直方图)的特征,也有一些文献采用了其他的特征,如\cite{bo2008fast}采用了SIFT(scale-invariant feature transform)和HistoSC(Histogram of Shape Context)的特征,\cite{eth_biwi_01027}采用了Lab颜色空间统计和皮肤检测的特征,本文同多数算法一样采用了HOG特征,并做了一些改进,此外,还提出了一种能表示皮肤信息的特征,取得了较好的结果。
\section{图片预处理}
为了剔除环境对提取特征的影响,在提取特征前首先要提取ROI(Region Of Interests)。在本文中,ROI即为图片中的人物,提取ROI即背景分离,任务是把人物从图片中分割出来,并裁剪图片。由图\ref{fig:demo}可以看出HumanEva数据库的环境是非常复杂的,好在数据库提供了纯背景的图像序列\ref{fig:background},因此可以在已知背景情况下分割人物。尽管这项任务是计算机视觉领域中一个非常久远且已经攻克的问题,但由于这不是本文重点,因此没有花太多时间和精力调查文献和实现算法,本文用一系列简单的操作实现了比\cite{Poppe2007}更好的分割效果。
\begin{figure}[htbp]
  \centering
  \subcaptionbox{C1}{\includegraphics[width=0.31\textwidth]{bk_C1}}
  \subcaptionbox{C2}{\includegraphics[width=0.31\textwidth]{bk_C2}}
  \subcaptionbox{C3}{\includegraphics[width=0.31\textwidth]{bk_C3}}
  \caption{HumanEva背景}\label{fig:background}
\end{figure}
\subsection{计算背景数据}
给出的背景不是单张图片,而是一段视频,因此可利用的信息更多。本文假定视频中的$N$帧图像每个像素点都满足高斯分布,记像素点$\mathbf{c}=(r,g,b)$,则$\mathbf{c}\sim \mathcal{N}(\mu,\sigma)$,通过统计所有视频帧可以算出$\mu$和$\sigma$,用待分割图片中的每一个像素点代入高斯分布,即可算出该点是否为背景的概率。给定一个阈值,即可产生一个粗糙的二值图分割,见图\ref{fig:bk1}。
\begin{figure}[htbp]
  \centering
  \subcaptionbox{原图}{\includegraphics[width=0.45\textwidth]{bkimage}}\hspace{.5cm}
  \subcaptionbox{背景概率分割}{\includegraphics[width=0.45\textwidth]{bk1}}
  \caption{HumanEva背景分割1}\label{fig:bk1}
\end{figure}

\subsection{局部优化}
可以看出,用简单的阈值分割无法保证正确性,部分背景被认为是前景(即所需人物),而前景中夹杂的与背景相似的颜色(尤其是脸部,和背景中的门颜色过于相近,残缺很多)则被认为是背景,于是又做了很多优化。首先,需要把明显不是前景的背景去掉,这里考虑用距离来度量,即偏离人物较远的均认为是背景。具体做法是先找到图中最大面积区域的位置,该区域被近似为前景位置,将偏离该区域较远的区域都认为是背景(因为数据库中人物的动作范围不大,长宽均有限,这里把阈值选为80 像素),可以从图\ref{fig:bkfar}看出偏远大块的背景已经被去掉。

\begin{figure}[htbp]
  \centering
  \subcaptionbox{处理前}{\includegraphics[height=5.5cm]{bkfar}}\hspace{1cm}
  \subcaptionbox{去掉偏远背景}{\includegraphics[height=5.5cm]{bkfarcut}}
  \caption{HumanEva背景分割2}\label{fig:bkfar}
\end{figure}

然而由于人物遮挡光线产生的阴影还存在,因此需要考虑去掉人物产生的阴影。去除阴影用了最简单的方法,即把原图先裁剪至最小前景区域,只考虑该区域底部20\%的区域,将这小部分区域转换到HSV颜色空间,同样在纯背景中对相同区域做HSV变换,仅对色相通道做差,手工选定阈值,再与已有分割对应部分做与运算,即可将阴影去除,阴影去除后的效果见图\ref{fig:bk2foot}。

\begin{figure}[htbp]
  \centering
  \subcaptionbox{原图}{\includegraphics[width=0.3\textwidth]{bkfootorg}}\hspace{.5cm}
  \subcaptionbox{背景}{\includegraphics[width=0.3\textwidth]{bkfootbk}}\\
  \subcaptionbox{原图HSV}{\includegraphics[width=0.3\textwidth]{bkfootorghsv}}\hspace{.5cm}
  \subcaptionbox{背景HSV}{\includegraphics[width=0.3\textwidth]{bkfootbkhsv}}\\
  \subcaptionbox{做差}{\includegraphics[width=0.3\textwidth]{bkfootdiff}}\\
  \caption{去除阴影}
\end{figure}

此时还有一些背景噪音,接着去掉了前景中面积较小的部分,面积的阈值可以选的小一些,在这里选择为40像素(图像为$480\times 640$像素),这些部分大多都是未能剔除的环境噪音,此时结果参见图\ref{fig:bk3small}。

\begin{figure}[htbp]
  \centering
  \subcaptionbox{去除阴影\label{fig:bk2foot}}{\includegraphics[height=5.5cm]{bkshadowrem}}\hspace{1cm}
  \subcaptionbox{去掉小块背景\label{fig:bk3small}}{\includegraphics[height=5.5cm]{bksmall1}}
  \caption{HumanEva背景分割3}
\end{figure}

由于人物的脸部等地方有较大缺失,于是接着做了8次膨胀\footnote{在膨胀前先对图像加框,防止越界}(半径为1的圆形,见图\ref{fig:bk4dilate}),和8次腐蚀(同样半径为1,见图\ref{fig:bk4erode}),最后做一次背景抹除(面积小于400像素的都被消去),此时只剩下了单一连通的前景。

\begin{figure}[htbp]
  \centering
  \subcaptionbox{膨胀\label{fig:bk4dilate}}{\includegraphics[height=5.5cm]{bkdilate}}\hspace{1cm}
  \subcaptionbox{腐蚀\label{fig:bk4erode}}{\includegraphics[height=5.5cm]{bkerode}}
  \caption{HumanEva背景分割4}
\end{figure}

与\cite{Poppe2007}对比可以看出,该文献的前景(见图\ref{fig:bk_cmp_org})有较多空洞,这对于HoG的计算会带来很大误差,而本文提出的分割方法(见图\ref{fig:bk_cmp_my})没有小块噪音,整个前景连为一体,虽然在边缘轮廓不是特别准确,但是对于HoG特征来说影响不大。

\begin{figure}[htbp]
  \centering
  \subcaptionbox{Poppe等人分割结果\label{fig:bk_cmp_org}}{\includegraphics[height=5.5cm]{bk_cmp_org}}\hspace{1cm}
  \subcaptionbox{本文结果1\label{fig:bk_cmp_my}}{\includegraphics[height=5.5cm]{bk_cmp_my}}\hspace{1cm}
  \subcaptionbox{本文结果2\label{fig:bk_cmp_my2}}{\includegraphics[height=5.5cm]{bk_final}}
  \caption{背景分割对比}
\end{figure}

此外,在处理演员2的时候由于衣服颜色和背景过于相像,出现了图\ref{fig:bkerr} 所示的错误(上身部分缺失),针对这种特殊情况,先找到前景所占矩形区域,然后填充矩形区域中上部60\%范围内的空洞。这一操作虽然不至于填充两腿缝隙,但对于胳膊与腰造成的空洞也会一并填充,鉴于这种情况出现的帧数极少,故忽略此问题。

\begin{figure}[htbp]
  \centering
  \includegraphics[height=5.5cm]{bkerr1}
  \caption{背景分割错误}\label{fig:bkerr}
\end{figure}

\section{HOG特征}
\subsection{概述}
HOG特征最早由Navneet Dalal和Bill Triggs在2005年提出~\cite{Dalal05histogramsof},被成功地用于目标检测,尤其是行人检测。该方法是对图像的局部区域出现的方向梯度进行统计,和SIFT相似,不同的是HOG用了重叠的局部对比度归一化(overlapping local contrast normalization)技术。HOG描述子的主要贡献是很好地描述了局部区域的边缘方向密度。对图像进行归一化后,能够很好地保证照射和阴影不变性。与其他方法相比,HOG有很好的几何和光学转化不变性,因此十分适合行人检测和姿态估计。
\subsection{计算HOG描述子}
HOG特征的提取方法主要步骤如下:
\begin{enumerate}
  \item 转化为灰度图像
  \item 划分成若干元胞(cell)
  \item 计算每个像素的梯度
  \item 统计每个子区域内梯度直方图
\end{enumerate}

具体说来,本文将$r,g,b$三个通道认为是三张灰度图,分别求取HOG特征,最后取每一维最大值,最后做归一化处理。伪代码见代码\ref{alg:maxhog}。
\newcommand{\comment}{\textbackslash\textbackslash}
\begin{algorithm}
  \caption{MAX-HOG}\label{alg:maxhog}
  \begin{algorithmic}[1]

    \FOR{$i=0$ to $image.NumberOfChannels$}
        \STATE H($i$) $\leftarrow$ GetHOG($image.channel(i)$);
    \ENDFOR

    \STATE  H $\leftarrow$ MAX(H($i$));
    \STATE  H $\leftarrow$ L1Norm(H);
    \RETURN H
  \end{algorithmic}
\end{algorithm}

针对每一张灰度图求取HOG特征的算法伪代码见代码\ref{alg:gethog}。

\begin{algorithm}
  \caption{GetHOG}\label{alg:gethog}
  \begin{algorithmic}[1]
    \STATE \comment 设置元胞数量,横向6个,纵向9个
    \STATE $Cell\_Num\_x \leftarrow 6;$
    \STATE $Cell\_Num\_y \leftarrow 9;$
    \STATE \comment 计算梯度\\
    \STATE $grad_x \leftarrow$ CalGrad$(image,hx)$;
    \STATE $grad_y \leftarrow$ CalGrad$(image,hy)$;
    \STATE $angle \leftarrow$ GetAngle$(grad_x,grad_y)$;\comment 计算梯度方向
    \STATE $magnit \leftarrow \sqrt{(grad_yu^2+grad_xr^2)}$;\comment 计算梯度大小
    \STATE \comment 计算直方图
    \FOR{$i=1$ to $Cell\_Num\_x\times Cell\_Num\_y$}
        \FOR{$j=0$ to 180, interval=20}
            \STATE H($i,j$) $\leftarrow \sum magnit(j \leq angle \leq j+$interval);
        \ENDFOR
    \ENDFOR
    \STATE  H $\leftarrow$ L1Norm(H);
    \RETURN H
  \end{algorithmic}
\end{algorithm}

此处需要强调一下,Poppe等人对于区域数量的选取是$5\times 6=30$,由于每张图片人物大小不一、长宽比例不固定,因此每个元胞的长宽比也不固定,这会导致一些问题,比如在不同长宽比的图片中,头部在细长的图片中会出现在多个元胞中,而较宽的图片中则只有一个区域有头部,图片本身的比例信息已经丢失,见图\ref{fig:broadthin}。为了解决这个问题,本文在求取HOG特征之前先将图片填充到3:2 的比例(见图\ref{fig:broadthin2}),这样求取的特征就不会变形,虽然有较多区域是无用信息,占用了维数,但是经过测试,这并不会增加额外的存储占用,因为数据本身是稀疏的。
\begin{figure}[htbp]
  \centering
  \subcaptionbox{较宽图片}{\includegraphics[height=5.5cm]{broad}}\hspace{.5cm}
  \subcaptionbox{较窄图片}{\includegraphics[height=5.5cm]{thin}}
  \caption{图片比例影响示意}\label{fig:broadthin}
\end{figure}

\begin{figure}[htbp]
  \centering
  \subcaptionbox{}{\includegraphics[height=5.5cm]{broad1}}\hspace{.5cm}
  \subcaptionbox{}{\includegraphics[height=5.5cm]{thin1}}
  \caption{图片固定比例示意}\label{fig:broadthin2}
\end{figure}

每个像素点根据周围像素信息都可以算得一个梯度信息,包括梯度的方向和大小。计算直方图指的是对于每一个元胞,对梯度的方向做直方图,方向范围是$0^\circ\thicksim180^\circ$,将直方图分为9块,每块含$20^\circ$,对在该角度范围的梯度的大小求和作为直方图中这一块的值。每个元胞的直方图统计可以用图\ref{fig:local_hoG}来表示,每一个方向用该方向上的矩形表示,颜色越亮表示梯度越大。最后做L1规范化,得到最终的HOG特征向量。

\begin{figure}[htbp]
  \centering
  \subcaptionbox{完整原图}{\includegraphics[height=5.5cm]{C1}}\hspace{.5cm}
  \subcaptionbox{右手腕局部}{\includegraphics[width=0.15\textwidth]{local_org}}\hspace{.5cm}
  \subcaptionbox{单个区域直方图\label{fig:local_hoG}}{\includegraphics[width=0.15\textwidth]{local_HoG}}
  \caption{HOG特征子区域直方图表示}
\end{figure}

\begin{figure}[htbp]
  \centering
  \subcaptionbox{C1}{\includegraphics[width=0.2\textwidth]{C1}}\hspace{.5cm}
  \subcaptionbox{C2}{\includegraphics[width=0.2\textwidth]{C2}}\hspace{.5cm}
  \subcaptionbox{C3}{\includegraphics[width=0.2\textwidth]{C3}}\\
  \subcaptionbox{HOG1}{\includegraphics[width=0.2\textwidth]{HOG1}}\hspace{.5cm}
  \subcaptionbox{HOG2}{\includegraphics[width=0.2\textwidth]{HOG2}}\hspace{.5cm}
  \subcaptionbox{HOG3}{\includegraphics[width=0.2\textwidth]{HOG3}}
  \caption{HOG特征}
\end{figure}

\section{皮肤特征}
由于人的皮肤是一个不变的特征,而且胳膊、头的位置对于姿态估计有重要作用,因此充分利用皮肤特征是十分必要的。对于姿态估计,我们只关心胳膊、头的位置,因此本文用位置、面积信息来描述皮肤特征。首先,需要判断每个像素属于皮肤的概率,为了简化问题,本文假定皮肤颜色在YCbCr颜色空间服从高斯分布,均值和方差均为手工给出。由于人们往往是穿着长裤,因此只能检测到头和左右臂这三个部分,而在穿长袖衣服时则只能检测到头和左右手。因此本文只选取概率最大的三个区域\footnote{只考虑ROI中的上半部分。有时候会小于三个区域,因为手和头部相连。},分别认为是头、左臂和右臂。按面积由大到小排序,每一个区域用$\mathbf{r}=(x,y,s)$表示,$(x,y)$为区域的重心坐标,$s$为区域面积。检测结果见图\ref{fig:skin}。
\begin{figure}[H]
  \centering
  \subcaptionbox{原图}{\includegraphics[width=0.3\textwidth]{skin}}\hspace{1cm}
  \subcaptionbox{皮肤检测}{\includegraphics[width=0.3\textwidth]{skinbw}}
  \caption{皮肤检测}\label{fig:skin}
\end{figure}

\section{特征表示}
\label{sec:feature}
HOG特征有$6\times 9=54$个元胞,每个子区域有9维直方图,因此共计$54\times 9=486$维。皮肤特征有3个区域,每个区域$(x,y,s)$三维信息,共计$3\times 3=9$维。两种特征相连共计$486+9=495$维。如果用到三个视角图片,则会有$495\times 3=1485$维。
